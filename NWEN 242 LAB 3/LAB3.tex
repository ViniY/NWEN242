\documentclass[]{article}
\usepackage{amsmath}
\usepackage[margin=0.7in]{geometry}
\usepackage{graphicx}
\usepackage{grffile}
\usepackage{graphics}
\usepackage{amssymb}% http://ctan.org/pkg/amssymb
\usepackage{pifont}% http://ctan.org/pkg/pifont
\newcommand{\cmark}{\ding{51}}%
\newcommand{\xmark}{\ding{55}}%
%opening
\title{NWEN242 LAB 3}
\author{Vincent Yu || 300390526}

\begin{document}
\maketitle
\section{4.1}
\subsection*{a}
R2  (v0) = 100004a0 \newline
R28 (gp) = 10008000\newline
R5  (a1) = 00000002 \newline
R6  (a2) = 0000000c\newline
R29 (sp) = 7fffe858\newline
\subsection*{b}
\begin{center}
	\begin{tabular}{c c c c c c}
		Set & V & LRU & Tag & Data Way0 & Acc \\
		0  & 1 & 0  & 400012 & 00000001 00000001  00000001 00000001& \\
		1  & 1 & 0  & 400012 & 00000002 00000002  00000002 00000002& hit\\  		
	\end{tabular}
\end{center}
DATA cache : Accesses: 8  Hits: 6 Hit Rate : 0.75000 \newline 
As the table shown above, we can find the value in $Array_A$ all store in the cache.\newline 
The cache we have is a 4 ways-associative and each block has the size of 16B. \newline
The cache hold 1,1,1,1 of array A in one block and hold 2,2,2,2 in an other block. In order to know how this works we need to convert 1 and 2 to binary numbers. 1=0000 0001 2=0000 0010 The index of set hold 1 is 0. The index of the set hold value" 2 " is set"1". \newline 
The index determined which block need to go so it can mapped the index with sets. And the offset determined which way need to go. If there is a hit, then it will push the whole block up to the processor.\newline


\subsection*{c}
\begin{center}
	\begin{tabular}{c c c c c }
		Set & V & Tag & Instruction & Acc \\
	    0   & 1 & 8000 &  lui \$1, 4096&\\
	    1   & 1 & 8000 &  ori \$2, \$1, 1152&\\
	    2   & 1 & 8000 &  lui \$1,4096&\\
	    3   & 1 & 8000 &  ori \$8, \$1, 1184&\\
	    4   & 1 & 8000 &  ori \$6, \$0, 0&\\ 
	    5   & 1 & 8000 &  ori \$4, \$0, 8&\\
	    6   & 1 & 8000 &  lw  \$5, 0(\$2)&\\
	    7   & 1 & 8000 &  lw  \$7, 0(\$8)&\\
	    8   & 1 & 8000 &  add  \$6, \$6, \$5&\\
	    9   & 1 & 8000 &  add  \$6, \$6, \$7&\\
	    10  & 1 & 8000 &  addi \$2, \$2, 4&\\
	    11  & 1 & 8000 &  addi \$8, \$8, 4&\\
	    12  & 1 & 8000 &  addi \$4, \$4, -1&\\
	    13  & 1 & 8000 &  slt \$1, \$0, \$4&\\
	    14  & 1 & 8000 &  bne \$1, \$0, -32&\\
	    15  & 1 & 8000 &  NULL&\\    
 	\end{tabular}
\end{center}
Because of display temporal locality, the most referenced instructions are hold in  cache. 
\subsection*{d}
\begin{center}
	\begin{tabular}{c c c c c c}
		Set & V & LRU & Tag & Data Way0 & Acc \\
		0  & 1 & 0  & 400012 & 00000001 00000001  00000001 00000001& \\
		1  & 1 & 0  & 400012 & 00000002 00000002  00000002 00000002& \\  
		2  & 1 & 0  & 400012 & 00000003 00000003  00000003 00000003& \\		
		3  & 1 & 0  & 400012 & 00000004 00000004  00000004 00000004&hit \\
	\end{tabular}
\end{center}
The idea of mapping is the same with b. But as shown above we have more sets hold value which means we will use more space in the cache to store all the values.
\section*{4.2}
\subsection*{a}
Block size 16B: Hit rate:0.75 \newline
Block Size 8B : Hit rate:0.5\newline
Block size 4B : Hit rate:0 \newline
Decreasing the block size, there is less data hold in every blocks. This will decrease the hit rate. For example if we put number 1 into the cache it automatically filled cache with numbers 1,2,3,4 (in binary version). If next instruction ask for 2 it will be a hit. But if we decrease the block size. And it only can hold one number in the block. The next instruction also want number 2. But this time it will be a miss, because the block doesn't hold the value.\newline
\subsection*{b}
N=1: Value in Register 6(sum result) :28(hexadecimal) 40(decimal) Hit Rate:0.75\newline
N=5: Value in Register 6(sum result) :c8(hexadecimal) 200(decimal) Hit Rate:0.95\newline
N=10: Value in Register 6(sum result) :190(hexadecimal) 400(decimal) Hit Rate:0.975\newline
N=100: Value in Register 6(sum result) :fa0(hexadecimal) 4000(decimal) Hit Rate:0.9975\newline
N=1000: Value in Register 6(sum result) :9c40(hexadecimal) 40000(decimal) Hit Rate:0.99975\newline
As N increasing the hit rate also increasing. I think this is because of the replace policy and temporal locality. We increase the iterations, it will increase the temporal locality. This will increase the hit rate.\newline
\section*{4.3}
\subsection*{a}
\begin{center}
\begin{tabular}{c c c c c c}
	Set & V & LRU & Tag & Instruction & Acc \\
	0 & 1 & 1 & 100001 & 00000041 00000042 000000043&\\
	\end{tabular}
\end{center}
The data cache store the values. But the cache we have it can only stores 64 numbers in the cache. So it follows LRU which means less recently used rule to replace . The data begins to overwrite the previous data once it is full. Because we are using direct mapping so it just start reimplementing from the beginning. \newline
\subsection*{b}
\begin{center}
	\begin{tabular}{c c c c c c c}
	     Way & Set & V & FIFO & Tag & Instruction & Acc \\
	     0   &	0 & 1 & 1 & 400006 & 00000060 00000061 000000062 00000063&\\
	     1   &  0 & 1 & 0 & 400007 & 00000070 00000071 000000072 00000073&\\
	\end{tabular}
\end{center}
As we decrease the cache size, it will increase the time of overwriting the data in the cache.\newline 
The first data entry into set 0 of the data cache is 60  which indicates the values have been overwritten 3 times. 
\subsection*{c}
\begin{center}
	\begin{tabular}{c c c c c c c}
		Way &Set & V & FIFO & Tag & Instruction & Acc \\
	     0&	0 & 1 & 1 & 400006 & 00000060 00000061 000000062 00000063&\\
	     1& 0 & 1 & 0 & 400007 & 00000070 00000071 000000072 00000073&\\
	\end{tabular}

\end{center}
FIFO is same with LRU. Because we use all the value in the cache , so the result is the same. If we only use a part of data in the cache. FIFO will replace the data from beginning, but LRU will replace the one has been used.\newline
\section*{4.4}
\begin{center}
	\begin{tabular}{c c c c c c}
		N &Stride & Hit rate & Misses & Hit rate(instruction) & misses\\
	1 & 1     &  0.75     & 32    &  0.980695    &  20\\
	1 & 2     &  0.5      & 32    &  0.961832    &  20\\
	1 & 4     &  0        & 32    &  0.925373    &  20\\
	1 & 8     &  0        & 16    &  0.857143    &  20\\
	1 & 16    &  0        &  8    &   0.736842   &20\\
	1 & 32    &  0        & 4     &  0.545455    &20 \\
	1 & 64    & 0         &2      &  0.285714    &20 \\
	5 & 1     & 0.75      &32     & 0.996118     &20 \\
	5 & 2     & 0.5       &32     & 0.992284     &20 \\
	5 & 4     & 0         & 32    &0.984756      & 20\\
	5 & 8     & 0         &16     & 0.970238     &20 \\
	5 & 16    & 0         &8      & 0.943182     & 20\\
	5 & 32    & 0.8        &4      & 0.895833     & 20\\
	5 & 64    & 0.8        &2      & 0.895833     & 20\\
	10& 1     & 0.75       &32    &0.998          &20\\
	10& 2     &0.5         &32    &0.996         &20\\
	10& 4     &0           &32    &0.992          &20\\
	10& 8     &0           &16    &0.97          &20\\
	10& 16    &0           &8      &0.97         &20\\
	10& 32    &0.9         &4     &0.94         &20\\
	10& 64    &0.9        &2      &0.90          &20\\
	100&1     &0.75       &32     &0.999        &20\\
	100&2     &0.5       &32     &0.999        &20\\
	100&4     &0         &32     &0.999        &20\\
	100&8     &0         &16     & 0.998        &20\\
	100&16    &0         &8     &0.997          &20\\
	100&32    &0.99      &4     &0.994          &20\\
	100&64    &0.99      &2     &0.990          &20\\
		
	\end{tabular}
\end{center}
Changing N can change the numbers of iterations. Changing stride value can change the step of using the numbers. For example stride is one then it will sum all the numbers up, but if stride is 4 it will only add 0 + 4 + 8....+ 127\newline
Because \$8 represent N and stride is the changing the step in the for loop.\newline
\section*{4.5}
\begin{center}
	\begin{tabular}{c c c c}
	Mapping type & Block Size &  Replacement & Hit Rate\\
	Direct Mapping & 4B    &LRU&	0.666667\\
	Direct Mapping & 8B    &LRU&    0.666667\\
	Direct Mapping & 16B    &LRU&    0.833333\\
	Direct Mapping & 4B    &FIFO&	0.666667\\
	Direct Mapping & 8B    &FIFO&    0.666667\\
	Direct Mapping & 16B    &FIFO&    0.833333\\
	2 Way          & 4B    &LRU&     0.666667\\
	2 Way          & 8B    &LRU&     0.666667\\
	2 Way          & 16B    &LRU&    0.833333\\
	2 Way          & 4B    &FIFO&     0.666667\\
	2 Way          & 8B    &FIFO&     0.666667\\
	2 Way          & 16B    &FIFO&     0.833333\\
	4 Way          & 4B    &LRU&     0.666667\\
	4 Way          & 8B    &LRU&     0.666667\\	
	4 Way          & 16B   &LRU&     0.833333\\
	4 Way          & 4B    &FIFO&     0.666667\\
	4 Way          & 8B    &FIFO&     0.666667\\	
	4 Way          & 16B   &FIFO&     0.833333\\
	Fully          & 4B    &LRU&     0.666667\\
	Fully          & 8B    &LRU&     0.666667\\
	Fully          & 16B    &LRU&     0.833333\\
    Fully          & 4B    &FIFO&     0.666667\\
    Fully          & 8B    &FIFO&     0.666667\\
    Fully          & 16B    &FIFO&     0.833333\\
	\end{tabular}
\end{center}

\end{document}
